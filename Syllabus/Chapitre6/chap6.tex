\ifx \globalmark \undefined %% This is default.
	\input{header}
	\begin{document} %% Crashes if put after (one of the many mysteries of LaTeX?).
\else 
	\documentclass{standalone}
	\begin{document}
\fi

\graphicspath{ {Chapitre6/images/} }

\setcounter{chapter}{5}
\chapter{Transformations d'état}
\chaptermark{Transformations d'état}\label{transetat}


\lettrine[lines=1]{\bf D}{}ans les chapitres qui précèdent, nous avons montré comment la démarche de modélisation peut être systématisée pour différentes classes de systèmes relevant de l'ingénierie. Pour chaque type de système, un modèle d'état général a été établi. Les variables d'état retenues dans ces modèles ont un sens physique précis~: positions et vitesses pour les systèmes mécaniques, courants et tensions pour les systèmes électriques, quantités totales pour les systèmes à compartiments, concentrations, volume et température pour les systèmes réactionnels. Il est cependant souvent utile pour analyser le comportement d'un système dynamique de procéder à une {\it transformation d'état} conduisant à un modèle équivalent du système mais exprimé dans de nouvelles variables d'état.

Outre les transformations d'état, il est aussi intéressant d'utiliser des représentations graphiques qui permettent de visualiser aisément certaines particularités structurelles du système. Parmi les représentations les plus courantes, on mentionnera le {\it schéma fonctionnel} et le {\it graphe du système} dont les définitions sont données ci-dessous.

\section{Schéma fonctionnel}

Le schéma fonctionnel d'un système dynamique est un graphe orienté 
dont chaque noeud est constitué par l'un des deux blocs fonctionnels 
représentés à la figure \ref{Fig:blocfonct}.
\begin{figure}[htbp] 
   \centering
   \includegraphics[width=10cm]{blocfonct} 
   \caption{Blocs fonctionnels : (a) intégrateur, (b) fonction}
   \label{Fig:blocfonct}
\end{figure}
\begin{itemize}
\item[$\bullet$] Le bloc fonctionnel Fig. \ref{Fig:blocfonct} 
(a) représente un intégrateur dont la variable d'entrée est 
la dérivée de la variable de sortie.
\item[$\bullet$] Le bloc fonctionnel Fig. \ref{Fig:blocfonct} 
(b) représente une fonction $f: \real^p \rightarrow \real$ dont la variable de 
sortie $z(t)$ est une fonction des variables d'entrée~:
\begin{equation*} \begin{split}
z(t) = f(x_1(t), x_2(t), \dots , x_p(t)).
\end{split} \end{equation*}
\end{itemize}
Dans certains cas, le dessin de ce bloc est particularisé de manière à rendre explicite la fonction qu'il représente. Trois exemples sont indiqués à la figure \ref{Fig:exblocfonct}.  Le schéma fonctionnel d'un système dynamique contient nécessairement $n$ intégrateurs dont les sorties sont les $n$ variables d'état du système. Ces intégrateurs sont interconnectés via des blocs fonctionnels représentant les différentes fonctions apparaissant dans les équations d'état. Les arcs du schéma fonctionnel s'interprêtent comme des lignes de transmission instantanée des variables qui leur sont attachées.
\begin{figure}[htbp] 
   \centering
   \includegraphics[height=55mm]{exblocfonct} 
   \caption{Exemples de blocs fonctionnels : (a) sommateur, (b) multiplieur, (c) produit par une constante}
   \label{Fig:exblocfonct}
\end{figure}

Outre leur intérêt pour l'analyse des systèmes dynamiques, les 
schémas fonctionnels constituent aussi un outil fondamental de 
programmation dans les langages standard de simulation dynamique 
tels que MATLAB/Simulink ou VisSim.

\begin{exemple}{\bf \em Des algues dans la lagune (suite)}

Au chapitre 5, nous avons établi un modèle simple décrivant la dynamique de croissance d'une population d'algues dans une lagune. En supposant que la cinétique de croissance obéit à une loi bilinéaire $r(x_1,x_2) = x_1x_2$, ce modèle s'écrit~:
\begin{equation*} \begin{split}
\dot x_1 &= -kx_1x_2 + u, \\
\dot x_2 &= x_1x_2 - dx_2.
\end{split} \end{equation*}
Le schéma fonctionnel correspondant est représenté à la figure 
\ref{Fig:schema}. \qed
\begin{figure}[htbp] 
   \centering
   \includegraphics[height=75mm]{schema} 
   \caption{Schéma fonctionnel du modèle de croissance d'algues}
   \label{Fig:schema}
\end{figure}
\end{exemple}

\section{Graphe d'un système dynamique}

Le graphe d'un système dynamique est, d'une certaine manière, le graphe complémentaire du schéma fonctionnel. En effet, ce sont les variables d'état $x_i$ et les variables d'entrée $u_j$ qui sont attachées aux noeuds du graphe tandis que les arcs (orientés) représentent les relations fonctionnelles entre ces variables.

Les règles de construction du graphe d'un système dynamique sont les suivantes~:
\begin{enumerate}
\item Le graphe contient $n+m$ noeuds étiquetés respectivement par les $n$ variables d'état $x_1, x_2, \dots , x_n$ et les $m$ variables d'entrée $u_1,u_2, \dots , u_m$.
\item Il y a un arc orienté de $x_i$ vers $x_j$ (ou de $u_k$ vers $x_j$) si la variable $x_i$ (ou $u_k$) apparait explicitement dans l'équation de la dérivée $\dot x_j$.
\end{enumerate}
\vv

\begin{exemple}{\bf \em Machine électrique à courant continu}

Considérons le modèle général d'une machine DC tel qu'il a été présenté au chapitre 3, section 3.5. C'est un système à quatre variables d'état et 3 variables d'entrée dont le modèle d'état s'écrit~:
\begin{equation*} \begin{split}
\dot x_1 &= x_2, \\
\dot x_2 &= J^{-1}( -h(x_2) + K_mx_3x_4 + u_3), \\
\dot x_3 &= L_f^{-1}(-R_fx_3 + u_1 ), \\
\dot x_4 &= L_a^{-1}(-R_ax_4 - K_ex_2x_3 + u_2).
\end{split} \end{equation*} 
\begin{figure}[htbp] 
   \centering
   \includegraphics[height=65mm]{grafmot} 
   \caption{Graphe du modèle d'état d'un moteur à courant continu}
   \label{Fig:grafmot}
\end{figure}
Le graphe de ce système est représenté à la figure \ref{Fig:grafmot}. \qed
\end{exemple}
Le graphe d'un système dynamique est un outil permettant de vérifier aisément si le système considéré possède des particularités structurelles intéressantes. Nous en verrons une illustration à la section \ref{triangulaire} lorsque nous étudierons les systèmes triangulaires.

\section{Transformations linéaires d'état}

Pour un système dynamique $\dot x = f(x,u)$, une transformation
linéaire d'état est une application linéaire $T : \real^n
\rightarrow \real^n$ bijective qui transforme l'état du système
$x \in \real^n$ en un nouvel état $z \in \real^n$ selon la règle:
\eqnn
z = Tx
\eeqnn
où $T$ est une matrice $(n \times n)$ régulière.

Dans les nouvelles coordonnées $z$, le modèle d'état du
système est transformé comme suit :
\begin{equation*} \begin{split}
\dot z = T \dot x = Tf(x,u)
\end{split} \end{equation*}
En exprimant que $x= T^{-1}z$ on obtient :
\eqnn
\dot z = g(z,u) \hd \text{avec} \hd g(z,u) \triangleq T f(T^{-1} z,u).
\eeqnn
En particulier, un modèle d'état linéaire $\dot x = Ax + Bu$
est transformé en un autre modèle linéaire :
\eqnn
\dot z = Fz + Gu \hd
\text{avec} \hd F \triangleq TAT^{-1}, \;\;   G \triangleq TB.
\eeqnn

\begin{exemple}{\bf \em Génératrice DC}

Au chapitre 3 (Section 3.6), nous avons établi le modèle d'état d'une
génératrice à courant continu. Lorsque la génératrice tourne à vitesse constante $\omega$, le modèle d'état est linéaire et s'écrit
\eqnn
\bpm
\dot x_1\\ \dot x_2 \epm = 
\bpm
- \dfrac{R_s}{L_s} & 0\\ & \vm \\ 
\dfrac{K_e \omega}{L_r} & - \dfrac{R_r + R_L}{L_r} \epm \bpm x_1\\x_2 \epm +
\bpm \dfrac{1}{L_s} \\ \vm \\ 0 \epm u
\eeqnn
où les variables d'état $x_1$ et $x_2$ représentent respectivement les
courants statorique et rotorique, tandis que l'entrée $u$ est la tension
appliquée au 
circuit statorique.

Nous définissons de nouvelles variables d'état $z_1$ et $z_2$ qui peuvent
être interprétées comme les flux magnétiques $\phi_s$ et $\phi_r$ auxquels
sont soumis
respectivement  les circuits statorique et rotorique :
\begin{equation*} \begin{split}
z_1 &= \phi_s = L_s x_1,\\
z_2 &= \phi_r =  L_r x_2 + K_e x_1.
\end{split} \end{equation*}
On observe qu'il s'agit bien d'une transformation d'état linéaire :
\eqnn
T = \bpm
L_s & 0\\ K_e & L_r \epm.
\eeqnn
La matrice $T$ est inversible (det $T = L_s L_r >0$) et la transformation
inverse s'écrit :
\eqnn
\bpm
x_1 \\x_2 \epm = \bpm \dfrac{1}{L_s} &  0\\ & \vm \\-\dfrac{K_e}{L_sL_r} &
\dfrac{1}{L_r} \epm \bpm z_1\\z_2 \epm.
\eeqnn
Dans les nouvelles coordonnées $(z_1, z_2)$, le modèle d'état s'écrit :
\begin{equation*} \begin{split} 
\bma{c} \dot z_1 \\ \dot z_2 \ema &= \bma{lcc}-\dfrac{R_s}{L_s} & & 0\\ & & \\
\dfrac{K_e\omega}{L_r} + \dfrac{K_e(R_r+R_L)}{L_r L_s}  - 
\dfrac{K_eR_s}{L^2_s} & & -\dfrac{R_r+R_L}{L_r} \ema \bma{c} z_1\\z_2 \ema \\ & \vspace{-4mm} \\ & \hd +
\bma{c} 1 \\ \vspace{-2mm} \\ \dfrac{K_e}{L_s} \ema u \hspace{8cm} \qed
\end{split} \end{equation*}
\end{exemple} 
\vv

\begin{exemple}{\bf \em Modèles linéaires à compartiments}

On s'intéresse ici aux modèles linéaires à compartiments tels que décrits à la section 4.4. Rappelons que la forme générale des équations d'état est la suivante~:
\eqnn
\dot x_i= \sum^n_{j=1} k_{ji} x_j - \sum^n_{\ell=0} k_{i\ell}x_i +b_iu_i, 
\hspace*{10mm} i = 1,n
\eeqnn
ou sous forme matricielle~:
\eqnn
\dot x = Ax + Bu
\eeqnn
avec $A$ une matrice de Metzler diagonalement dominante et $x_i$ la quantité totale contenue dans le compartiment $i$.

On souhaite exprimer le modèle en termes de concentrations.  On introduit les notations :
\begin{equation*} \begin{split}
V_i &: \mbox{ volume du compartiment } i,\\
a_{ij} &\triangleq  k_{ij} V_i,\\
z_i&= \frac{x_i}{V_i} : \mbox{ concentration dans le compartiment } i.
\end{split} \end{equation*}
A l'aide de ces notations, on peut réécrire le modèle comme suit :
\begin{equation*} \begin{split}
\dot x_i &= \sum^n_{j=1} \frac{a_{ji}}{V_j} x_j - \sum^n_{l=0}\frac{a_{il}}{V_i}x_i + b_i u_i,\\
\dot x_i&= \sum^n_{j=1} a_{ji} z_j -\sum_{\ell=0}^n a_{i\ell} z_i + b_i u_i,
\end{split} \end{equation*}
et donc :
\eqnn
\dot z_i = \sum^n_{j=1} \frac{a_{ji}}{V_i} z_j - \sum^n_{\ell=0} \frac{a_{i\ell}}{V_i} z_i + \frac{b_i}{V_i} u_i
\eeqnn
On a opéré ainsi une transformation d'état en passant des quantités  totales $x_i$ aux concentrations $z_i$ comme variables d'état.  Sous forme matricielle la transformation d'état s'écrit :
\eqnn
z = V^{-1}x \mbox{ avec } V \triangleq \mbox{ diag} \{ V_i, i=1, \ldots, n \}
\eeqnn
Dans les coordonnées de concentration, le modèle devient :
\eqnn
\dot z = Fz + Gu
\eeqnn
avec $ F \triangleq V^{-1} AV$ et $ G \triangleq V^{-1} B$.  
On peut vérifier que la matrice $F^T$ est aussi une matrice de Metzler diagonalement dominante.  \qed
\end{exemple}
\vv

\begin{exemple}{\bf \em Diagonalisation et constantes temps}

On considère un modèle linéaire $\dot x = Ax +Bu$ dont la matrice $A$ a
 toutes ses valeurs propres $\lambda_i$ réelles, distinctes et non-nulles.  
 Elle est alors diagonalisable, c'est à dire qu'il existe une matrice
  $T$ telle que
\eqnn
D \triangleq TAT^{-1} = \mbox{ diag}(\lambda_i, i=1, n)
\eeqnn
Si on définit une transformation d'état :
\eqnn
z = Tx
\eeqnn
\noindent le système est transformé en :
\eqnn
\dot z = Dz + TBu
\eeqnn
ou, encore composante par composante :
\eqnn
 \dot z_i = \lambda_i z_i + \beta_i u \;\;\;\; i=1,n
\eeqnn
où $\beta_{i}$ est la i-ème ligne de la matrice $TB$. Les grandeurs
$\tau_i = |\lambda_i|^{-1}, i=1,\ldots , n$, sont les {\em constantes de temps du système}.
\begin{figure}[htbp]    \centering
   \includegraphics[height=10cm]{schemadiag} 
   \caption{Schéma fonctionnel d'un système diagonalisé à une entrée}
   \label{Fig:schemadiag}
\end{figure}

We have now operate a state transformation going from total quantities  $x_i$ to concentrations $z_i$ as state variables. In a matrix format the state transformation can be written as :
\eqnn
z = V^{-1}x \mbox{ avec } V \triangleq \mbox{ diag} \{ V_i, i=1, \ldots, n \}
\eeqnn
In the concentration coordinates, the model becomes :
\eqnn
\dot z = Fz + Gu
\eeqnn
with $ F \triangleq V^{-1} AV$ et $ G \triangleq V^{-1} B$.  
We can check that the matrix $F^T$ is also a diagonally dominant Metzler matrix.  \qed
\end{exemple}
\vv

\begin{exemple}{\bf \em Diagonalization and time constants}

Let us consider a linear model $\dot x = Ax +Bu$ in which the eigen values $\lambda_i$ of the matrix $A$ are real, distinct and non-zero. Then the matrix $A$ is diagonalizable, which means there is a matrix $T$ such that
\eqnn
D \triangleq TAT^{-1} = \mbox{ diag}(\lambda_i, i=1, n)
\eeqnn
If we define a state transformation :
\eqnn
z = Tx
\eeqnn
\noindent The system is then transformed into :
\eqnn
\dot z = Dz + TBu
\eeqnn
or, components by components : 
\eqnn
 \dot z_i = \lambda_i z_i + \beta_i u \;\;\;\; i=1,n
\eeqnn
where $\beta_{i}$ is the i-th row of the matrix $TB$. The parameters
$\tau_i = |\lambda_i|^{-1}, i=1,\ldots , n$, are the {\em time constants of this system}.
\begin{figure}[htbp]    \centering
   \includegraphics[height=10cm]{schemadiag} 
   \caption{Block diagram of a diagonalized system with one entry}
   \label{Fig:schemadiag}
\end{figure}

We have now replaced the initial model in which the state variables can be strongly related by a collection of first-order systems completely separated from each other as we can see on the block diagram illustrated in figure \ref{Fig:schemadiag}.

For instance let us consider a DC motor controlled by the stator (see chapter 3, section 3.6) with $h(\omega) = B\omega$~:
\eqnn
\dfrac{d}{dt} \bma{c} I_s\\ \omega\ema = \bma{cc}-\dfrac{R_s}{L_s}& 0\\ & \vm \\
\dfrac{K_mI_r}{J} & -\dfrac{B}{J} \ema  \bma{c} I_s\\ \omega\ema +
\bma{c}\dfrac{1}{L_s}u_1\\ \vm \\ \dfrac{1}{J} u_2 \ema,
\eeqnn
we check that time constants are  
\begin{equation*} \begin{split}
 \tau_e &= \dfrac{L_s}{R_s}  \text{ electric time constant},\\
\tau_m &= \dfrac{J}{B} \text{ mechanical time constant}. \xqedhere{3.5cm}{\qed} 
\end{split} \end{equation*}
\end{exemple}
\vv

\begin{exemple}{\bf \em Reaction systems as multi-compartment systems}

In chapter 5, we have seen that the state model for reaction systems can be written as
\eqnn 
\dot x =  Cr(x) + q_{in}(x,u) - q_{out}(x,u).  
\eeqnn
Let us introduce the following notations for entry and output vectors :
\begin{equation*} \begin{split}
q_{in}(x,u) &\teq \Big(q_{o1}(x,u), q_{o2}(x,u), \hdots , q_{on}(x,u)\Big)^T, \\
q_{out}(x,u) &\teq \Big(q_{1o}(x,u), q_{2o}(x,u), \hdots , q_{no}(x,u)\Big)^T.
\end{split} \end{equation*}
Let us now suppose that the system is conservative and that the flows $q_{oi}$ and $q_{io}$ meet the conditions C1, C2 and C3 form chapter 4.
Then the reaction system is equivalent to a multi-compartments system with the following linear state transformation :
\eqnn
z = T x, \hd T \teq \textrm{ diag} \{ \omega_2, \omega_2, \hdots , \omega_n \}.
\eeqnn
To illustrate that property, let us consider again the example of the perfectly mixed chemical reactor (see example 5.6). In that reactor, the two reactions 
\begin{equation} \begin{split} \label{exa}
X_1 + X_2 \; &\longrightarrow \; 2X_3, \\ 
2X_3 \; &\longrightarrow \; X_4 
\end{split} \end{equation}
occur simultaneously in the liquid phase with kinetics 
\begin{equation} \begin{split} 
r_1(x) &= k_1x_1x_2e^{-(Kx_4)}, \\
r_2(x) &= k_2x_3^2. 
\end{split} \end{equation}
The reactor is fed by the two initial reactives $X_1$ and $X_2$ in solution with feeding concentrations $x_1^{in}$ and $x_2^{in}$.

The state model can be written as 
\eqnn
\bpm \dot x_1 \\ \dot x_2 \\ \dot x_3 \\ \dot x_4 \epm =
\bpm -1 & 0 \\ -1 & 0 \\ 2 & -2 \\ 0 & 1 \epm \bpm
k_1x_1x_2e^{-(Kx_4)} \\ k_2x_3^2 \epm 
 + u \bpm x_1^{in} -x_1 \\ x_2^{in} -x_2 \\ -x_3 \\ -x_4
\epm   
\eeqnn
where the state variables $x_1, x_2, x_3$ and $x_4$ represents the species concentrations in the reaction medium.

We can easily check that the system is conservative with the normalization vector $\omega = (1,1,1, 2)$. Thus, we define the linear state transformation 
\eqnn
z_1 = x_1, \hd z_2 = x_2, \hd z_3 = x_3, \hd z_4 = 2x_4.
\eeqnn
In those new coordinates, we obtain a multi-compartments system of which the graph is illustrated in figure \ref{reacompart} and of which the state model is :
\eqnn
\bpm \dot z_1 \\ \dot z_2 \\ \dot z_3 \\ \dot z_4 \epm = \bpm -k_1 z_2 \varphi - u & 0 & 0 & 0 \\ 0 & -k_1 z_1 \varphi - u & 0 & 0 \\ k_1 z_2 \varphi & k_1 z_1 \varphi & -2 k_2 z_3 - u & 0 \\ 0 & 0 & 2 k_2 z_3 & -u \epm \bpm z_1 \\ z_2 \\ z_3 \\ z_4 \epm + \bpm u x_1^{in} \\ u x_2^{in} \\ 0 \\ 0 \epm 
\eeqnn
with
\eqnn
 \varphi \teq \exp(- \dfrac{K}{2} z_4). \xqedhere{5cm}{\qed}
\eeqnn

\end{exemple}
\begin{figure}[tbp] 
   \centering
   \includegraphics[height=30mm]{reacompart} 
   \caption{Compartment representation of a reaction system}
   \label{reacompart}
\end{figure}

\section{Non-linear state transformations}

For a non-linear state model $\dot x = f(x,u)$, it is often more interesting to consider non-linear state transformations. However it is generally not possible to define {\em global} transformations that are valid for every $x \in \real^n$. We are then interested in {\em local} transformations that are defined in a susbset of $\real^n$.

\begin{definition}{\bf \em Non-linear state transformation}

Let $U$ and $V$ be two open subset of $\real^n$. A non-linear state transformation is an application  $T : U \rightarrow V$ that transforms the state of the system $x \in U$ in a new state $z \in V$ :
$$
z = T(x)
$$
and that possess the following properties :
\begin{enumerate}
\item[a)] the application $T$ is bijective, which means that there is an inverse function $T^{-1} : V \rightarrow U$ such that $x = T^{-1} (z)$,
\item[b)] $T(x)$ ans $T^{-1}(z)$ are functions of class $C^{1}$, that is to say continuous and differentiable.
\end{enumerate}
\noindent The state transformation is said to be {\em global} if $U=V=\real^n$. \qed
\end{definition}

A transformation $T$ possessing those properties is called a diffeomorphism. Its bijectivity is necessary to reverse the state variables change and to go back to the initial state variables.
The property b)  ($T$ and $T^{-1}$ are of classes $C^1$) is necessary to express the state model in the new coordinates as follows :
$$
\dot z = \frac{\partial T}{\partial x} \dot x = \frac{\partial
T}{\partial x} f(x, u)
$$
where, by using $x = T^{-1}(z)$, we obtain
$$
\dot z = g(z,u)
$$
with :
\begin{equation*} \begin{split} 
g(z,u) \triangleq \left[ \frac{\partial T}{\partial x}
f(x,u)\right ]_{x = T^{-1}(z)}.
\end{split} \end{equation*}
In a similar way, we can express :
$$
f(x,u) \triangleq \left[\frac{\partial T^{-1}}{\partial z}
g(z,u)\right ]_{z = T(x)}
$$
The properties given in the following lemma can be useful to demonstrate the existence of a non-linear state transformation. 
\begin{lemme}{\blanc}
\begin{enumerate}
\item If the jacobian matrix $[\partial T/\partial x]$ is non-singular at $x_0$, then, by the inverse funtction theorem, there is a neighborhood $U$ around $x_0$ such that the application $T$ restricted to $U$ is a diffeomorphism on $U$.
\item $T$ is a global diffeomorphism if and only if~:
\begin{itemize}
\item[a)] $[\partial T/\partial x]$ is non-singular for every $x$ in $\real^n$;
\item[b)] $\lim_{\|x\|\rightarrow\infty}\|T(x)\| = \infty$. \qed
\end{itemize}
\end{enumerate}
\end{lemme}

\section{Mechanical systems}

As we have seen in chapter 2, the state vector of a mechanical system is made of two parts : the position coordinates $q$ and the speed coordinates
$v=\dot q$
$$ x = \bma{c} q \\ v \ema. $$
In numerous applications, it is interesting to consider different sets of position coordinates. The state transformation is then made through two steps. First we transform the position coordinates :
$$ p = \phi (q)$$
where $\phi : U_1 \rightarrow V_1$ is a diffeomorphism and 
$\partial \phi /\partial q$ is of full rank $\forall q \in U_1$.

The new state vector is then formed by the new position coordinates $p$ and their derivatives $w = \dot p$ :
$$
z = \bma{c} p\\w \ema.
$$
The state transformation is then defined as follows :
\begin{equation*} \begin{split}
 z = T(x), \hspace*{10mm} \bma{c}p\\w  \ema = \bma{c} \phi(q)\\
\dfrac{\partial \phi}{\partial q} v \ema.
\end{split} \end{equation*}
The inverse state transformation is :
$$
x=T^{-1}(z), \hd \bma{c}q\\v \ema = \bma{l} \phi^{-1}(p) \\
\left(\dfrac{\partial \phi}{\partial q}\right )^{-1}_{q = \phi^{-1} p}w
\ema.
$$

\begin{exemple}{\bf \em Polar and cartesian coordinates}

In the method described in chapter 2 to establish a state model for articulate mechanical systems, the position of each body's center of mass is given by its cartesian coordinates $q =(x,y)$, as showed on the figure \ref{Fig:cocarpol}.
\begin{figure}[htbp] 
   \centering
   \includegraphics[width=6cm]{cocarpol} 
   \caption{Cartesian coordinates and polar coordinates}
   \label{Fig:cocarpol}
\end{figure}
Another set of frequently used coordinates are the polar coordinates $r$ and $\alpha$ : $r$ is the distance between the origin and the center of mass and $\alpha$ is the angle between the axis $OX_b$ and the vector $\small \overrightarrow{OG}$.

The transformation allowing to go from cartesian coordinates to polar coordinates can be written as :
\begin{equation*} \begin{split}
&q = \bma{c} x \\y \ema \;\;\; p = \bma{c} r\\ \alpha \ema,\\
& \vm \\
&p = \phi(q) : \left\{ \begin{array}{ll}
r =&\sqrt{x^2 + y^2},\\ 
\alpha  =&\mbox{arc} \sin \dfrac{y}{\sqrt{x^2 + y^2}} 
\end{array}
\right..
\end{split} \end{equation*} 
The inverse transformation $q = \phi^{-1} (p) $ is written :
\begin{equation*} \begin{split}
x &= r \cos \alpha,\\
y &= r \sin \alpha.
\end{split} \end{equation*}
We notice that the change of coordinates $p = \phi(q)$ is not defined at the origin, that is when $x=0$ and $y=0$.  We also check taht
$$ \det[\frac{\partial \phi^{-1}}{\partial p} ] = r
$$
is zero when $r=0$ (which is the origin also). It follows that the tranformation of coordinates is not global but only valid on the following sets :
\begin{equation*} \begin{split}
U_1 &=\real^2 \backslash \{(0,0)\},\\
V_1 &= \real^2 \backslash \{(r, \alpha) : r = 0\}.
\end{split} \end{equation*}
Finally, the complete state transformation between the state 
$(q,v)$ and the state $(p,w)$ is written as follows :
\begin{equation*} \begin{split}
r&= \sqrt{x^2 +y^2},\\
\alpha &= \mbox{arc} \sin \frac{y}{\sqrt{x^2 +y^2}},\\
\dot r &= \frac{x \dot x + y \dot y}{\sqrt{x^2 +y^2}},\\
\dot \alpha &= \frac{x \dot y - \dot x y}{x^2 + y^2},
\end{split} \end{equation*}
and the inverse transformation :
\begin{equation*} \begin{split}
x&= r \cos \alpha,\\
y &= r \sin \alpha,\\
\dot x &= \dot r \cos \alpha - r \dot \alpha \sin \alpha,\\
\dot y &= \dot r \sin \alpha + r \dot \alpha \cos \alpha. \xqedhere{4.6cm}{\qed}
\end{split} \end{equation*}
\end{exemple}

\begin{exemple}{\bf \em Articular coordinates and robotic task coordinates}

For manipulator robots consisting of as many actuators as degrees of freedom, with rotoïd joints, articular coordinates from chapter 2 are \gf natural \gf coordinates for the description of the system : every coordinate gives the position of an arm compared to the previous one.
Generally, with those coordinates, the model becomes quite simple. The articular models are adequate for the conception of systems designed to control robots.

On the point of view of the user, interested for instance by planning trajectories, the task coordinates, that is the coordinates of the effector, are more interesting. Let us consider for instance a planar robot with two degrees of freedom moving in a horizontal plane (see figure \ref{Fig:robplan}).
\begin{figure}[htbp]
   \centering
   \includegraphics[width=6cm]{robplan} 
   \caption{Articular coordinates and task coordinate for a robot with two degrees of freedom.}
   \label{Fig:robplan}
\end{figure}
The articular coordinates are the angles $\alpha_1$ and $\alpha_2$, the tasks coordinates are the cartesian coordinates $X$ and $Y$.
Then we have : 
$q = (\alpha_1, \alpha_2)$ et $p = \phi(q) = (X,Y)$.  The transformation allowing to go from articular coordinates to task coordinates are written 
\eqn
 X &=& l_1 \cos \alpha_1 + l_2 \cos(\alpha_1 + \alpha_2)\label{artache1},\\
 Y &= &l_1 \sin\alpha_1 + l_2 \sin(\alpha_1 + \alpha_2)\label{artache}.
\eeqn
\noindent We easily check that this transformation can not be injective : a position $(X,Y)$ of the effector corresponds to two distinct and symmetrical positions of the robot. To correctly define a transformation of coordinates, we must define the domains $U$ and $V$ corresponding to the application $\phi$ and its inverse.

We first observe that the image of the application $\phi$ is necessarily restricted to the circle of accessible positions for the robot, that is (if $l_2 > l_1)$ a circle of radius $l_1 + l_2$ :
$$ 
V_1 \triangleq \{(X,Y) : (l_2 - l_1)^2 < X^2 + Y^2 < (l_1 + l_2)^2 \}.
$$
On the other hand, the domain of $\phi$ must be chosen so the application is injective. A possible choice is the following :
$$
U_1 \triangleq \{(\alpha_1, \alpha_2) : -\pi < \alpha_1 < \pi \;\;\; 0 <
\alpha_2 < \pi\}.
$$
With those definitions, we can check that the application 
$$
\phi : U \longrightarrow V
$$
defined by the equations (\ref{artache1})-(\ref{artache}) is a diffeomorphism.

Then we need to complete the transformation to extend it to the speed coordinates. The state vectors written in articular coordinates and in task coordinates are defined as follows : 
$$
x^T = (\alpha_1, \alpha_2, \dot \alpha_1, \dot\alpha_2), \hspace*{10mm} z^T
= (X,Y,\dot X, \dot Y).
$$
The state transformation $z = T(x)$ can finally be written as :
\begin{equation*} \begin{split}
X &= l_1 \cos\alpha_1 + l_2 \cos(\alpha_1 + \alpha_2),\\
Y&= l_1 \sin\alpha_1 + l_2 \sin(\alpha_1 + \alpha_2,\\
\dot X &=-l_1 \dot \alpha_1\sin\alpha_1 -l_2 \dot \alpha_1\sin(\alpha_1+\alpha_2) - l_2 \dot \alpha_2\sin(\alpha_1 + \alpha_2),\\
\dot Y &= l_1 \dot \alpha_1\cos\alpha_1 + l_2 \dot \alpha_1\cos(\alpha_1+\alpha_2) + l_2 \dot \alpha_2\cos(\alpha_1 + \alpha_2). \xqedhere{1.8cm}{\qed}
\end{split} \end{equation*}
\end{exemple}

\section{Electrical machines}

In chapter 3, we have obtained a general model for rotating electrical machines of the form :
\begin{equation*} \begin{split}
L(\theta) \dot I &= -\omega K(\theta) I - RI +V, \\
\dot\theta &= \omega, \label{machel}\\
J \dot \omega &= \frac{1}{2}I^TK(\theta)I - h(\omega) + T_a, 
\end{split} \end{equation*}
with
$$
K(\theta)\triangleq\frac{\partial L(\theta)}{\partial \theta}.
$$
These equations naturally lead to establish state models in which the state vector
$$
x^T = (I^T, \theta, \omega)
$$
is made of currents $I$, angular position $\theta$ and angular speed $\omega$.  Other choices of state variables can be used to ease the mathematical study of electrical machines. A current transformation consists of replacing currents by flows :
$$
\phi = L(\theta)I,
$$
that is transforming the state vector $x^T = (I^T, \theta, \omega)$ into the state vector $z^T = (\phi^T, \theta, \omega)$. This transformation is actually a diffeomorphism because the inductance matrix 
$L(\theta)$ is invertible for all $\theta$.

In the new state variables $z$, the equations (\ref{machel}) can be written : 
\begin{equation*} \begin{split}
\dot\phi &= -RL^{-1}(\theta)\phi+ V,\\
\dot \theta &= \omega,\\
J\dot\omega &= \dfrac{1}{2} \phi^T G(\theta) \phi - h(\omega) + T_a,\\
\mbox{ avec } G(\theta) &\triangleq  L^{-1}(\theta)K(\theta)L^{-1}(\theta).
\end{split} \end{equation*}
\vv

\section{Triangular systems} \label{triangulaire}
 
 A system with {\it only one} entry (mono-entry system)
 \eqn
 \dot x = f(x,u) \hh \hh x \in \real^n \hh \hh u \in \real \label{mono}
 \eeqn
is said to be {\it triangular} if it meets the following definition.

\begin{definition}{\bf \emph Triangular system}

A dynamic mono-entry system is triangular if there is a state variable $x_i$ such that the shortest path from $u$ to $x_i$ in the system's graph is of length $n$. \qed
\end{definition}

For a triangular system, it is therefore always possible to renumber the state variable such that the state model can be written as :
\begin{equation} \begin{split} \label{systriang}
\dot x_1 &= g_1(x_1,x_2),  \\
\dot x_2 &= g_2(x_1,x_2, x_3),  \\
&\vdots  \\
\dot x_i &= g_i(x_1,x_2, \dots ,x_{i+1}),  \\
&\vdots  \\
\dot x_{n-1} &= g_{n-1}(x_1,x_2, \dots ,x_n),  \\
\dot x_n &= g_n(x_1,x_2, \dots ,x_n,u).  
\end{split} \end{equation}
We observe that the number of state variables on the right increases progressively from $2$ to $n$ (which is why it's called a triangular form). Besides, the entry $u$ appears only in the last equation.
\vv

\begin{exemple}{\bf \em Manipulative robot with one degree of freedom and an elastic joint \label{exrobot}}

The state model of such a robot with an elastic rotoid joint and negligeable friction torques can be written as:
\begin{equation} \begin{split} \label{robotelast}
\dot x_1 &= x_2,  \\
J_1\dot x_2 &= -mgd\sin x_1 - k(x_1 - x_3),  \\
\dot x_3 &= x_4,  \\
J_2\dot x_4 &= k(x_1 - x_3) + u.
\end{split} \end{equation}
where

$x_1$ is the angular position coordinate of the arm,

$x_2$ is the angular speed of the arm,

$x_3$ is the angular position coordinate of the motor,

$x_4$ is the angular speed of the motor,

$J_1$ and $J_2$ are the inertia momentum of the arm and the motor,

$d$ is the distance between the joint and the center of mass,

$k$ is the elastic spring constant ,

$u$ is the commanded torque developped by the motor.\\

\noindent The graph of the system is represented on figure
\ref{Fig:grafrobot} and we can check that the state equations have the wanted triangular structure. \qed
\begin{figure}[htbp]
   \centering
   \includegraphics[width=9cm]{grafrobot} 
   \caption{Graph of the model of a single arm robot with an elastic joint.}
   \label{Fig:grafrobot}
\end{figure}
\end{exemple}

\section{Brunovski canonical form}\label{sectionbrunovski}

\begin{definition}{\blanc}

A dynamic mono-entry system (\ref{mono}) can be written under the Brunovski canonical form if there is a state transformation $ T : U \rightarrow V$ and an open interval $W \subset \real$ 
  such that, in the new state variables $z=T(x)$, the system takes on the following particular triangular form:
 \begin{equation*} \begin{split}
 \dot z_1 &= z_2,\\
 \dot z_2 &= z_3,\\
 \vdots &\\
 \dot z_n &= \alpha (z_1, z_2, \ldots, z_n, u),
\end{split} \end{equation*}
where the function $\alpha$ is continuous and invertible according to $u$ over $W$ for all $z \in V$.
\qed
\end{definition}

 We observe that the system is therefore made of a chain of integrators such that
 \begin{equation*} \begin{split}
 \dot z_i = z_{i+1}\;\;\; i = 1. \ldots, n-1
 \end{split} \end{equation*}
 and that all system non-linearities are focused on the only non-linear scalar function $\alpha (z_1, z_2, \ldots, z_n, u)$.  The Brunovski canonical form can also be schematized as indicated on figure \ref{Fig:bruno}. The Brunovski form is interesting because it allows to easily plan trajectories as we will see in chapter 10.
  \begin{figure}[htbp] 
    \centering
    \includegraphics[width=9cm]{bruno} 
    \caption{Functionnal schem of Brunovski cannonical form}
    \label{Fig:bruno}
 \end{figure}
 
\begin{exemple}{\bf \em A chemical reactor \label{exreachim} }
 
Let us consider a perfectly mixed continuous reactor with constant volume in which occurs an irreversible chemical reaction using two species $X_1$ and $X_2$ :
 $$
 X_1 \longrightarrow X_2.
 $$
 The reactor is only fuelled with $X_1$, with constant concentration
 $c$.  The input variable is the specific volumetric input flow rate of the reactor. The kinetics obeys the law of mass action. According to the modelisation principles established in chapter 5, we get a bilinear state model :
 \begin{equation*} \begin{split}
 \dot x_1 &= -kx_1 + u (c-x_1),\\
 \dot x_2 &= kx_1 - ux_2.
 \end{split} \end{equation*}
 We define the following state transformation $z = T(x)$ :
 \begin{equation*} \begin{split}
 z_1 &= \frac{x_2}{c-x_1},\\
 z_2 &= \frac{kx_1(c-x_1-x_2)}{(c-x_1)^2}.
 \end{split} \end{equation*}
 The domain $U$ and the image $V$ of the application $T : U \longrightarrow V$
 are defined according to :
 \begin{equation*} \begin{split}
 U &= \{(x_1, x_2) : x_1 > 0, x_2 > 0, x_1+x_2 < c \},\\
 V &= \{(z_1,z_2) : 0<z_1 <1, z_2>0 \}.
 \end{split} \end{equation*}
We can then show that the state transformation $z = T(x)$ hereby defined is a diffeomorphism and its inverse is :
 \begin{equation*} \begin{split} 
 x_1 &= \frac{cz_2}{k(1-z_1)+z_2},\\
 x_2 &= \frac{ckz_1(1-z_1)}{k(1-z_1)+z_2}.
 \end{split} \end{equation*}
 In the new coordinates, the state model is under Brunovski cannonical form :
 \begin{equation*} \begin{split} 
 \dot z_1 &= z_2,\\
 \dot z_2 &= -\left( z_2 + \dfrac{(k+1)z^2_2}{k(1-z_1)} \right) + (k(1-z_1) + z_2)u.
 \end{split} \end{equation*} 
The function $\alpha$ is invertible according to $u$ over $W$. \qed
\end{exemple}

This example shows that it is difficult to determine \textit{a priori} if a given dynamic system can be put under Brunovski form and to find the appropriate state transform. However, if the system is already given in a triangular form, here is a sufficient condition to put it in the Brunovski form :
\begin{lemme}{\blanc} \label{lemmeici}

A triangular dynamic system  described by the state model (\ref{systriang}) can be put under Brunovski cannonical form around $(x_0,u_0)$ if the following inegalities :
\begin{equation*} \begin{split}
\frac{\partial g_i}{\partial x_{i+1}} &\neq 0 \hh \hh i=1, \ldots, n-1, \\
\frac{\partial g_n}{\partial u} &\neq 0,
\end{split} \end{equation*}
are satisfied in $(x_0,u_0)$.
\qed
\end{lemme}
\vv

\begin{exemple}{\bf \em Manipulative robot with one degree of freedom and an elastic joint (continued)}

Let us consider again the model (\ref{robotelast}) of the example \ref{exrobot}. We easily check that the conditions of the Lemma \ref{lemmeici} are satisfied for all $x \in \real^4$ and naturally lead to the state transformation :
\begin{equation*} \begin{split} 
z_1 &= x_1 \\
z_2 &= x_2 \\
z_3 &= -J_1^{-1}[mgd\sin x_1 + k(x_1 -x_3)] \\
z_4 &= -J_2^{-1}[mgdx_2 \cos x_1 + k(x_2 - x_4)].
\end{split} \end{equation*}
The inverse state transformation is written as :
\begin{equation*} \begin{split} 
x_1 &= z_1 \\
x_2 &= z_2 \\
x_3 &= (mgdk^{-1} \sin z_1 + z_1 + J_1k^{-1}z_3) \\
x_4 &= (mgdk^{-1} z_2 \cos z_1 + z_2 + J_2k^{-1}z_4) 
\end{split} \end{equation*}

On observe qu'il s'agit d'un difféomorphisme global de $\real^4$ dans $\real^4$. Avec les nouvelles variables d'état le modèle s'écrit sous forme de Brunovski~:
\begin{equation*} \begin{split} 
\dot z_1 &= z_2 \\
\dot z_2 &= z_3 \\
\dot z_3 &= z_4 \\
\dot z_4 &= J_2^{-1}[mgd(z_2^2 \sin z_1 - z_3 \cos z_1) -k z_3] \\ & \hh \hh + kJ_2^{-2}[mgd\sin z_1 + J_1z_3 - u]
\end{split} \end{equation*}
On observe aussi que la fonction $\alpha$ est inconditionnellement inversible sur $\real$ par rapport à $u$. La forme de Brunovski est donc ici globalement valide. \qed
\end{exemple}
Pour des systèmes qui ne sont pas donnés sous forme triangulaire mais qui sont affines en l'entrée, le lemme suivant exprime des conditions utiles pour trouver la transformation d'état.
\begin{lemme}{\blanc}
Un système affine en l'entrée
$$
\dot x = f(x) + g(x)u \hh \hh x \in \real^n \hh \hh u \in \real
$$
peut être mis sous forme canonique de Brunovski dans un domaine $U \subset \real^n$ si il existe une transformation d'état $z = T(x)$ vérifiant les conditions suivantes~:
\begin{equation*} \begin{split} 
& T_{i+1}(x) = \frac{\partial T_i}{\partial x}f(x) \hh \hh i=1,2, \ldots, n-1, \\
& \frac{\partial T_i}{\partial x}g(x) = 0 \hh \hh i=1,2, \ldots, n-1, \\
& \frac{\partial T_n}{\partial x}g(x) \neq 0,
\end{split} \end{equation*}
pout tout $x \in U$. \qed
\end{lemme}
\vv

\begin{exemple}{\bf \em Un réacteur chimique (suite)}

Nous montrons comment utiliser le lemme précédent pour retrouver la transformation d'état qui a été postulée sans justification dans l'exemple \ref{exreachim}. Le modèle d'état s'écrit~:
\begin{equation*} \begin{split}
\bma{c} \dot x_1 \\ \dot x_2 \ema = \bma{c} -x_1 \\ x_1 \ema + \bma{c} c-x_1 \\ -x_2 \ema u \triangleq f(x) + g(x)u
\end{split} \end{equation*}
On considère tout d'abord l'équation aux dérivées partielles~:
\begin{equation*} \begin{split}
\frac{\partial T_1}{\partial x}g(x)=0 \hh \Rightarrow \hh \frac{\partial T_1}{\partial x_1}(c-x_1) = \frac{\partial T_1}{\partial x_2}x_2
\end{split} \end{equation*}
dont une solution est~:
\begin{equation*} \begin{split}
T_1(x) = \frac{x_2}{c-x_1}
\end{split} \end{equation*}
On calcule ensuite~:
\begin{equation*} \begin{split}
T_2(x) = \frac{\partial T_1}{\partial x}f(x) \hh \Rightarrow \hh T_2(x) = \frac{kx_1(c-x_1-x_2)}{(c-x_1)^2}
\end{split} \end{equation*}
On détermine le domaine $U$ et l'image $V$ de l'application $T:U \rightarrow V$ ainsi définie. On vérifie enfin que la condition $(\partial T_2/\partial x)g(x) \neq 0$ est satisfaite sur $U$~:
\begin{equation*} \begin{split}
\frac{\partial T_2}{\partial x}g(x) = \frac{c(c-x_1-x_2)}{(c-x_1)^2} \neq 0 \xqedhere{3.9cm}{\qed}
\end{split} \end{equation*}
\end{exemple}

\section{Exercices}

\begin{exercice}{\bf \em Un four de verrerie}

Au Chapitre 1, Exemple 1.1 et Exercice 1.1, nous avons proposé trois jeux différents de variables d'état pour un modèle de four de verrerie. Déterminer les trois transformations d'état correspondantes et indiquer leurs domaines de définition. \qed
\end{exercice}
\vv


\begin{exercice}{\bf \em Un relais electromagnétique}

Soit le relais électromagnétique dont le modèle d'état à été
établi au Chapitre 3, Exemple 3.2.
\begin{enumerate}
\item On choisit les nouvelles variables d'état suivantes : $y_1 = z$,
$y_2 = \dot z$, $y_3 = \phi(I,z)$. Montrer qu'il s'agit d'une
transformation d'état valide. Etablir le modèle d'état dans ces
nouvelles variables.
\item Montrer que le système peut être mis sous forme canonique de
Brunovski. Déterminer la transformation d'état et donner une
interprétation physique des nouvelles variables d'état. \qed
\end{enumerate}
\end{exercice}
\vv

\begin{exercice}{\bf \em Une cage d'ascenseur}

\begin{tabular}{p{6.5cm}p{2mm}c}
\vspace{-3.2cm} 
Sur la figure ci-contre, on a représenté une cage d'ascenseur suspendue à un cable élastique de masse négligeable.
\vspace{2mm}

\noindent {\bf Notations :}

$y$ = longueur du câble

$\omega$ = vitesse angulaire de la poulie

$R$ = rayon de la poulie

$m$ = masse de la cage

\vspace{2mm}
\noindent La tension dans le câble est modélisée par la loi de Hooke :
\begin{equation*}
T = \frac{k(y - z)}{z}
\end{equation*}
où $z$ est une variable d'état auxiliaire dont la dérivée est la vitesse périphérique de la poulie : $\dot z = R \omega$.
& &
\parbox[c]{6cm}
{\includegraphics[width=4.5cm]{ascenseur.pdf}}
\end{tabular}

\begin{enumerate}
\item Etablir un modèle d'état avec 4 variables d'état : $y, \dot y, z, \omega$. Le frottement est négligé. La variable d'entrée est le couple de rotation $u$ appliqué à la poulie.
\item Montrer que le système peut être mis sous forme de Brunovski. Expliciter la transformation d'état. \qed
\end{enumerate}
\end{exercice}
\vv

\begin{exercice}{\bf \em Des coccinelles et des pucerons}

Montrer qu'il existe une transformation d'état telle que le système (\ref{coc}) du chapitre 1 modélisant l'interaction entre les populations de coccinelles et de pucerons peut être mis sous la forme d'un système à compartiments. Dessiner le graphe associé. Déterminer les flux $q_{ij}$, la matrice $L$ et la matrice $A(x,u)$. \qed
\end{exercice}
\vv

\begin{exercice}{{\bf \em Un réacteur biochimique}}

Soit un réacteur continu parfaitement mélangé et à volume
constant dans lequel se déroule une réaction chimique autocatalytique
irréversible mettant en oeuvre deux espèces $A$ et $B$~:
\e
A + B \longrightarrow 2B
\ee
Le réacteur est alimenté uniquement avec l'espèce $A$, à
concentration constante. La variable d'entrée est le débit
volumétrique d'alimentation du réacteur. Les cinétiques obéissent
à la loi d'action des masses. 
\begin{enumerate}
\item Etablir les équations d'état du système.
\item Montrer que le système est conservatif.
\item Déterminer une transformation d'état qui mette le
système sous forme canonique de Brunowski.
\item Déterminer la transformation d'état qui met le système sous la forme d'un système à compartiments.
\item Mêmes questions si la réaction est réversible. \qed
\end{enumerate}
\end{exercice}
\vv

\begin{exercice}{\bf Un four électrique}

Un four électrique est chauffé par une thermistance comme indiqué sur la figure ci-dessous.
\begin{figure}[h]
\begin{center}
\includegraphics[width=8cm]{four}
\label{four}
\end{center}
\vspace{-5mm}
\end{figure}
\begin{enumerate}
\item Etablir un modèle d'état du système sous les hypothèses de modélisation suivantes :
\begin{itemize}
\item[a)] La thermistance est une résistance dont la valeur varie avec la temperature suivant la relation de Reinhart-Hart :
\begin{equation*}
\dfrac{1}{T} = a + b \ln R + c (\ln R)^3
\end{equation*}
où $a, b, c$ sont des constantes caratéristiques positives fournies par le constructeur.
\item [b)] Comme représenté sur la figure, la thermistance est alimentée par une batterie de tension constante $E$ via une inductance (linéaire) constante et une résistance (linéaire) règlable qui est l'entrée du système.
\item[c)] Le four est chauffé par la thermistance. La perte de chaleur à travers les parois du four est proportionnelle à la différence  entre la température à l'intérieur du four et la température extérieure qui est supposée constante. 
\end{itemize}
\item Montrer que le système peut être mis sous forme de Brunovski. Expliciter la transformation d'état. \qed
\end{enumerate}
\end{exercice}
\vv

\begin{exercice}{\bf \em Un système à deux compartiments}

Soit le système {\em linéaire} à deux compartiments dont le graphe
est indiqué à la figure \ref{Fig:deuxcomp}. Déterminer la transformation d'état qui diagonalise le
système. Expliciter les constantes de temps. \qed
\begin{figure}[h] 
   \centering
   \includegraphics[width=6cm]{deuxcomp} 
   \caption{Graphe d'un système à deux compartiments}
   \label{Fig:deuxcomp}
\end{figure}
\end{exercice}

\end{document}
