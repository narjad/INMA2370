\ifx \globalmark \undefined %% This is default.
	\input{header}
	\begin{document} %% Crashes if put after (one of the many mysteries of LaTeX?).
\else 
	\documentclass{standalone}
	\begin{document}
\fi

\graphicspath{ {Chapitre2/images/} }


\setcounter{chapter}{1}
\chapter{Articulated mechanical systems}
\chaptermark{Articulated mechanical systems}\label{systmeca}




\lettrine[lines=1]{\bf T}{}he subject of this chapter is setting in equations state models of mechanical systems formed of a set of rigid bodies connected by joints. The systematic modeling method that we will study is applicable to many practical examples of mechanical systems such as vehicles (cars, trains, planes, ...) or robots. This method results from a systematic application of Newton's law.

In order to simplify the notations and calculations, we will limit ourselves to the establishment of the motion equations in a two-dimensional space
(i.e. in a plane). The extension to the case of a movement in a three-dimensional space is conceptually simple but more difficult to visualize.

We first consider the case of a single rigid body without friction. Then, we discuss the modeling of an articulated system consisting of several rigid bodies. The modeling method is described in detail using an example of a \blue{robot manipulator} with two degrees of freedom. Finally, we discuss how to extend the model to take into account friction, the elasticity of the joints and the non-holonomic constraints.

\section{Dynamics of a rigid body in the plane}

We consider a rigid body moving in a plane in which an orthonormal inertial basis $ 0,X_B, Y_b$ is fixed arbitrarily (Fig.\ref{Fig:corigidplan}).
\begin{figure}[h]
\begin{center}
\includegraphics[width=6cm]{corigidplan}
\caption{Coordinates of a rigid body in plane}
\label{Fig:corigidplan}
\end{center}
\end{figure}
A vector $\vec{W}$ is attached to the body. The position of the body is completely specified by the 3 coordinates $x, y, \theta$: 
\begin{itemize}
\item $x, y$ are the Cartesian coordinates of the center of mass $G$ in the fixed basis $0,X_b,Y_b$~;
\item $\theta$ is the orientation of the vector $\vec{W}$ with respect to the fixed basis $0,X_b,Y_b$. 
\end{itemize}
We define the three-dimensional vector describing the position of the body:
\eqn
q \triangleq \bma{c} x \\y \\ \theta \ema. \label{cogen}
\eeqn

\noindent A direct application of Newton's laws, coordinate by coordinate, then leads to the following general equations of motion:
\begin{itemize}
\item Equations of translation of the center of mass :
\eqnn
m\ddot{x} &=& F_x, \\ m\ddot{y} &=& F_y.
\eeqnn
\item Equation of rotation around the center of mass :
\eqnn
I\ddot{\theta} = T.
\eeqnn
\end{itemize}

\noindent where $m$ is the mass of the body, $I$ is its moment of inertia with respect to the center
mass,
$F_x$ and
$F_y$ denotes the projections of the resultant of the forces applied to the body on the
axes $0X_b$ and $0Y_b$ 
respectively and $T$ is the resultant of the torques applied
for the rotation of the body around the center of mass.

These general equations of motion form the basis of the establishment
of the system state model as we will illustrate it in an example.

\begin{exemple}{\bf Modeling of the dynamics of a rocket.}

We consider a rocket moving in a plane perpendicular to the ground. This rocket is propelled by two jet engines arranged symmetrically with respect to the body of the rocket as shown in Figure \ref{Fig:fusee}. 
\begin{figure}[ht]
\begin{center}
\includegraphics[width=8cm]{fusee}\hfill
\includegraphics[width=4.6cm]{ariane}
\caption{Modeling of the dynamics of a rocket - Photo of the Ariane rocket during takeoff (\copyright \,ESA)}
\label{Fig:fusee}
\end{center}
\end{figure}
The equations of motion are established under  {\bf the modeling assumption} that the rocket is a rigid body of constant mass.
\begin{itemize}
\item Equations of translation :
\begin{equation} \begin{split} \label{transfus}
m\ddot{x} &= F_x = (F_1 + F_2)\cos\theta, \\ 
m\ddot{y} &= F_y = (F_1 + F_2)\sin\theta - mg_0.  
\end{split} \end{equation}
\item Equation of rotation :
\eqn
I\ddot{\theta} = T = (F_2 - F_1)d\sin\alpha. \label{rotfus}
\eeqn
\end{itemize}
\noindent 
In these equations, $(x, y)$ is the position of the center of mass $G$, $\theta$ the angle of the vector $\vec W$ with respect to the horizontal, $F_ {1}, F_ {2}$ \blue{the thrusts of the reactors}, $m$ the mass of the rocket, $I$ its moment of inertia, $d, \alpha$ geometrical parameters (Fig.\ref{Fig:fusee}) and $g_0$ the gravitational constant.

The equations (\ref{transfus})-(\ref{rotfus}) can be put in the standard form of a state model $\dot{x} =
f(x,u)$ of dimension 6 with two inputs if we introduce the following notations~:
\begin{description}
\item {\em State variables:}
\eqnn
x_1 = x, \hspace{6mm} x_2 = y, \hspace{6mm} x_3 = \theta, \hspace{6mm} x_4 = \dot{x}, \hspace{6mm} 
x_5 = \dot{y}, \hspace{6mm} x_6 = \dot{\theta}.
\eeqnn
\item {\em Input variables :}
\eqnn
u_1 = F_1, \hspace{10mm} u_2 = F_2.
\eeqnn
\end{description}
The state model is written as follows~:
\begin{equation*} \begin{split}
\dot x_1 &= x_4, \\
\dot x_2 &= x_5, \\
\dot x_3 &= x_6, \\
\dot x_4 &= \frac{\cos x_3}{m} (u_1 + u_2), \\
\dot x_5 &= - g_0 + \frac{\sin x_3}{m} (u_1 + u_2), \\
\dot x_6 &= \frac{d \sin \alpha }{I} (u_2 - u_1). \qed
\end{split} \end{equation*}
\end{exemple}

A special situation appears when the considered body is subject to a set of forces whose resultant is zero but which are not all applied at the same point. The equations of motion can be written as :
\eqnn
m\ddot{x} &=& 0\\ 
m\ddot{y} &=& 0 \\
I\ddot{\theta} &=& T 
\eeqnn
In such a case, it is common, in some applications, to not specify the forces that are behind the torque $T$, but to directly consider this one as the cause of the movement. We say, for simplicity, that the body is subjected to a {\em torque}. Thus, for example, we will talk about the torque provided by an engine to rotate a \blue{robot manipulator} segment.

The state model obtained in the example of the rocket is non-linear with respect to the state variables and affine in the input variables. This will be the case for most applications of interest for which the translational and rotational equations describing the dynamics of a rigid body can be written in the general matrix form :
\eqnn
J\ddot{q} + b(q) = B(q)u.
\eeqnn
In this equation $J$ is the inertia matrix (diagonal and constant), $b(q)$
represents the effect of gravity and 
$B(q)$ is a
matrix (called kinematic) non-linearly depending on the state variables. 
We deduce that the state model is written in the following general form :
\eqnn
\dot{q} &=& v, \\ 
\dot{v} &=& J^{-1}[-b(q) + B(q)u],
\eeqnn
where $v \triangleq \dot{q}$ is called {\em vector of generalized velocities}.

\section{Dynamics of articulated mechanical systems}

We now consider the case of an articulated mechanical system with N body. The general procedure for setting in equations the state model can be summarized as follows~:
\begin{enumerate}
\item Fix an inertial reference frame in the system configuration space and $N$ moving frames attached to the centers of mass of the $N$ bodies of the system.
\item Write the \blue{path and bonding constraints} equations faced by the movement of the system. Deduce the number of degrees of freedom.
\item Write the equations of motion (translation and rotation) for each of the coordinates by including the bonding forces related to the constraints (method of Lagrange coefficients).
\item Remove the coefficients of Lagrange and redundant coordinates.
\end{enumerate}

We will now detail the procedure, explaining the new concepts (degrees of freedom, Lagrange coefficients, redundant coordinates) that have been mentioned, and illustrate it with a typical example: the development of the dynamic model of a \blue{robot manipulator} with two degrees of freedom.

\vspace{5mm}

\noindent {\bf First step: Defining coordinate}

An inertial reference frame is set in the configuration space $\Omega$ of the system. $N$ moving frames are attached to the centers of mass of the $N$ bodies of the system. The position of the system is at any time characterized by the coordinate vector
\eqnn
\xi = (x_1 \; y_1 \; \theta_1 \; . \; . \; .\; x_N\; y_N\; \theta_N)^{T}
\eeqnn
of dimension $3N$.

\vspace{5mm}

\noindent {\bf Step Two: Expression of geometric constraints}

The movement of an articulated mechanical system may be subject to two types of constraints (called geometric) : \blue{path contraints} on the one hand and bonding contraints between the bodies on the other hand. These constraints are expressed as a set of algebraic relations between the coordinates which we will note
\eqnn
\Psi (\xi) = 0,
\eeqnn
where $\Psi$ is an application $\Omega \rightarrow \RR^p$ of class $C^1$ and $p$
denotes the number of constraints. According to the implicit function theorem, in a neighborhood of any point $\xi$
of the configuration space, there is a partition $\xi = (q, \bar{q})$ of the coordinates vector such that :
\begin{itemize}
\item the dimension (noted $\sigma$) of $\bar{q}$ is equal to the rank of the Jacobian matrix of the application $\Psi$ :
\eqnn
\sigma \triangleq \mbox{dim}\bar{q} = \mbox{rank}\frac{\partial \Psi}{\partial \xi};
\eeqnn
\item we can express the coordinates $\bar{q}$ depending on coordinates $q$ :
\eqn
\bar{q} = \phi (q) \label{red}.
\eeqn
\end{itemize}
\noindent This means that we can use the expression (\ref{red}) to remove the {\em redundant coordinates} $\bar q$ of the system description. The size of the vector $q$ of the coordinates that are preserved is the {\em number of degrees of freedom} of the system, denoted $\delta$ :
\eqnn
\delta \triangleq 3N - \sigma.
\eeqnn

\vspace {5mm}

\noindent {\bf Step Three: Equations of motion}

Then we write the equations of motion (translation and rotation) for each of the coordinates by including the bonding forces related to the constraints. The partition $(q,\bar{q})$ of the coordinates induces a similar partition of the set of equations of motion as follows :
\eqn
J\ddot{q} + b(q,\bar{q}) &=&  B(q,\bar{q})u + w, \label{mo1}\\[2mm]
\bar{J}\ddot{\bar{q}} + \bar{b}(q,\bar{q}) &=& \bar{B}(q,\bar{q})u + \bar{w}. \label{mo2}
\eeqn
In these equations, vectors $w$ and  $\bar{w}$ represent the bonding forces that ensure that the constraints are satisfied at any time during the movement of the system. It is shown in the mechanical basic works that these binding forces are expressed as follows :
\eqnn
w &=& - A(q)\lambda, \\
\bar{w} &=& \lambda.
\eeqnn
where $\lambda$ is the vector of {\em Lagrange coefficients} (of dimension $\sigma$)
and $A(q)$ is the matrix of dimension $\delta \times \sigma$ defined as follows :
\eqnn
A(q) \triangleq (\frac{\partial \phi}{\partial q})^{T}.
\eeqnn

\vspace{5mm}

\noindent {\bf Step Four: Eliminate redundant coordinates}

In the equation (\ref{mo2}), $\lambda$ is expressed as :
\eqnn
\lambda = \bar{J}\ddot{\bar{q}} + \bar{b}(q,\bar{q}) - \bar{B}(q,\bar{q})u.
\eeqnn
By substituting this expression  in  (\ref{mo1}) and using (\ref{red}), it is resulted as :
\begin{equation} \begin{split}
J\ddot{q} + A(q)\bar{J}\ddot{\bar{q}} &+ b(q,\phi(q)) + A(q)\bar{b}(q,\phi(q)) \\
&= (B(q,\phi(q)) + A(q)\bar{B}(q,\phi(q)))u \label{g1}.
\end{split} \end{equation}
Last step to do is the elimination of $\ddot {\bar q}$. To obtain this product, it is needed to differentiate two times the expression (\ref{red}) :
\eqn
\dot {\bar q} &=& A^{T}(q)\dot q \\
\ddot {\bar q}  &=& A^{T}(q)\ddot q + \dot A^T(q)\dot q. \label{ac}
\eeqn
Substituting this last expression  (\ref{ac}) in (\ref{g1}) and introducing these following notations:
\begin{equation*} \begin{split}
M(q) &\triangleq J + A(q)\bar{J}A^{T}(q), \\
f(q,\dot{q}) &\triangleq A(q)\bar{J}\dot{A}^{T}(q)\dot{q}, \\
g(q) &\triangleq b(q,\phi(q)) + A(q)\bar{b}(q,\phi(q)), \\
G(q) &\triangleq B(q,\phi(q)) + A(q)\bar{B}(q,\phi(q)),
\end{split} \end{equation*}
The general dynamic model of an articulated mechanical system is finally obtained under this form :
\eqn
M(q)\ddot q + f(q,\dot q) + g(q) = G(q)u. \label{modmecgen}
\eeqn
\noindent In this equation :
\begin{itemize}
\item $q$ is the vector (of dimension $\delta$) of coordinates  necessary for the description of the systeme,
\item $M(q)$ is the inertia matrix (of dimensions $\delta \times \delta$) symmetrical and defined positive,
\item $f(q,\dot{q})$ is the vector (of dimension $\delta$) that represents forces and couples resulting  of relative liaisons  to constraints; it could also be written as 
\eqnn
f(q,\dot{q}) = C(q,\dot{q})\dot{q}
\eeqnn
where $C(q,\dot{q})$ is the matrix of dimensions $\delta \times \delta$ defined below :
\eqnn
C(q,\dot{q}) \triangleq A(q)\bar{J}\dot{A}^{T}(q),
\eeqnn
\item $g(q)$ is a vector (of dimension $\delta$) representing forces and couples which resulting of gravity,
\item $u$ is a vector (of dimension $m$) of forces and couples applied to the system ,
\item $G(q)$ is a cinematic matrix of dimensions  $\delta \times m$.
\end{itemize}

\noindent Once the general dynamic model  (\ref{modmecgen}) established, the last step remaining is the deduction of the state model of the system :
\begin{equation*} \begin{split}
\dot{q} &= v, \\ 
\dot{v} &= M^{-1}(q)[-f(q,v) - g(q) + G(q)u].
\end{split} \end{equation*}
In these state equations, $q$ is the coordinates vector of position and 
$v=\dot{q}$ is the coordinates vector of speed.

\begin{exemple} {\bf Dynamic model of a manipulator robot .}

A manipulator robot is formed by a set of rigid articulated segments. There is two kinds of joints : the revolute joint and the prismatic joint. A revolute allows a relative movement of rotation between two segments. A prismatic joint allows a relative movement of translation between two segments. 

Robots are operated by recessed engines producing translation forces for prismatic joints and rotation couples for revolute joints.

Let’s consider the manipulator robot represented in the figure 
\ref{Fig:robot} formed by rigid segment 
moving horizontally (body 1) which a second rigid segment is articulated. The second segment could perform a rotation movement (body 2). The movement of the system is due to the application of the force $F$ applied horizontally on the first segment and the couple of rotation $T$ applied on the second segment. The inertial basis and the different coordinates are indicated on the figure. 


\begin{figure}[ht]
\begin{center}
\includegraphics[width=8.5cm]{robot2}
\caption{Modelisation of a manipulator robot}
\label{Fig:robot}
\end{center}
\end{figure}

This system is subjected to these constraints  :
\begin{description}
\item {\em Course  constraints  :}
\begin{equation*} \begin{split}
y_1 &= 0, \\
\theta_1 &= 0.
\end{split} \end{equation*}
\item {\em Liaison constraints   :}
\begin{equation*} \begin{split}
x_2 - b\sin\theta_2 - x_1 - a&= 0, \\
y_2 + b\cos\theta_2 &= 0.
\end{split} \end{equation*}
\end{description}
\noindent The course constraints show the fact that the body 1 can only have a horizontal movement. The liaison constraints express the relation existing between the Cartesian coordinates of the centers of mass of these two bodies due to their joint. The Jacobian matrix of constraints is written as followed :
$$
\bma{cccccc} 0 & 1 & 0 & 0 & 0 & 0 \\ 
0 & 0 & 1 & 0 & 0 & 0\\ -1 & 0 & 0 & 1 & 0 & -b\cos\theta_2 \\
0 & 0 & 0 & 0 & 1 & -b\sin\theta_2 \ema.
$$
It is observed that this matrix is a full row matrix  $\sigma = 4$ and so the system has $\delta = 2$ degrees of freedom ( as it is expected ). It is also observed that the partition $(q,\bar{q})$ can be described with this coordinates like:
$$
q = \bma{c} x_1 \\ \theta_2 \ema, \hh 
\bar{q} = \bma{c} y_1 \\ \theta_1 \\ x_2 \\ y_2\ema.
$$
It is easy to verify that in all the configuration space, coordinates 
$\bar{q}$ can be expressed as an explicit function  $\bar{q} = \phi(q)$ of coordinates  $q$~:
\begin{align}
y_1 &= 0, \label{c1} \\
\theta_1 &= 0, \label{c2} \\
x_2 &= x_1 + b\sin\theta_2 + a, \label{c3} \\
y_2 &= - b\cos\theta_2. \label{c4}
\end{align}
It is now possible to eliminate the coordinates  $\bar{q}=(y_1, \theta_1, x_2, y_2)^{T}$ of the description of the system and just keep the coordinates $q=(x_1, \theta_2)^{T}$.
The matrix $A(q)$ is written :
$$  
A(q) = (\frac{\partial \phi}{\partial q})^{T} = \bma{cccc} 0 & 0 & 1 & 0 \\ 
0 & 0 & b\cos\theta_2 & b\sin\theta_2 \ema
$$  
The movement equations are written  :
\begin{align}
m_1\ddot x_1 &= F - \lambda_3, \label{eqmo1}\\
I_2\ddot \theta_2 &= - \lambda_3b\cos\theta_2 - \lambda_4b\sin\theta_2 
+ T, \label{eqmo2} \\ 
m_1\ddot y_1 &= - m_1g_0 + \lambda_1, \label{eqmo3}\\
I_1\ddot \theta_1 &= \lambda_2, \label{eqmo4}\\
m_2\ddot x_2 &= \lambda_3, \label{eqmo5}\\
m_2\ddot y_2 &= -m_2g_0 + \lambda_4. \label{eqmo6}
\end{align}
By combining constraints  (\ref{c1}), (\ref{c2}) ) with the movement equations
(\ref{eqmo3}), (\ref{eqmo4}) the following values  $\lambda_1$ et $\lambda_2$are deducted :
$$
\lambda_1 = m_1g_0,\hd \lambda_2 =0.
$$
These values express the liaison forces applied on the two bodies to satisfy the course constraints along the movement system. 

Moreover, by suppressing  $\lambda_3$ et $\lambda_4$ between the movement equations
(\ref{eqmo1}), (\ref{eqmo2}), (\ref{eqmo5}), (\ref{eqmo6}), this result is obtained:
\eqn
\bma{c} m_1\ddot x_1 \\ I_2\ddot \theta_2 \ema + \bma{cc} 1 & 0 \\ b\cos \theta_2
& b\sin \theta_2 \ema \bma{c} m_2\ddot x_2 \\ m_2\ddot y_2 \ema = 
\bma{c} F \\ T - bm_2g_0\sin\theta_2 \ema. \nonumber \\ \label{t1} 
\eeqn
By derivating two times the constraints  (\ref{c3}), (\ref{c4}), the result is obtained :
\eqn
\bma{c} m_2\ddot x_2 \\ m_2 \ddot y_2 \ema = \bma{cc} 1 & b\cos\theta_2 \\ 0 & 
b\sin\theta_2 \ema \bma{c} m_2\ddot x_1 \\ m_2\ddot \theta_2 \ema + 
m_2b {\dot \theta_2}^2 \bma{c} -\sin\theta_2 \\ \cos\theta_2 \ema. \label{t2}
\eeqn
By substituting  (\ref{t2}) in (\ref{t1}), the model of the system is obtained under the desired form:
\eqn
M(q)\ddot{q} + C(q,\dot{q})\dot{q} + g(q) = G(q)u,  \label{eqmouv}
\eeqn
with
\begin{equation*} \begin{split}
M(q) &= \bma{cc} m_1 + m_2 & m_2b\cos\theta_2 \\ m_2b\cos\theta_2 & I_2 + m_2b^2 \ema, \\
C(q,\dot{q}) &= \bma{cc} 0 & -m_2b\dot \theta_2 \sin\theta_2 \\ 0 & 0 \ema, \\
g(q) &= \bma{c} 0 \\ bm_2g_0\sin\theta_2 \ema, \\
G(q)u &= \bma{c} F \\ T \ema. \end{split} \end{equation*}
\qed

\end{exemple}

\section{Properties of the inertia matrix }

\begin{enumerate}
\item The inertia matrix $M(q)$ is symmetrical and defined positive. Indeed, this matrix is the sum of a diagonal matrix $J$ whose the elements are positive and a matrix symmetrical and semi-defined positive $A(q)\bar{J}A^{T}(q)$.
\item The temporal derivative of the inertia matrix$\dot M(q)$ verifies this following relation:
\begin{equation*} \begin{split}
\dot M(q) &= A(q)\bar{J}\dot A^{T}(q) + \dot A(q)\bar{J}A^{T}(q),  \\
&= C(q,\dot q) + C^T(q,\dot q). \label{prop2}
\end{split} \end{equation*}
This relation involves that the matrix 
\eqn
\dot M(q) - 2C(q,\dot q) \label{prop1}
\eeqn
is anti-symmetric.
\item 	The inertia matrix $M(q)$ verifies the following relation :
\eqn
\frac{\partial}{\partial q}(\dot q^T M(q) \dot q) = \dot q^T C(q, \dot q). \label{prop2}
\eeqn
The verification of this expression is left as an exercise.
\end{enumerate}


\section{Elastic joints}

Until this moment, it has always been considered that articulated mechanic systems were formed only by rigid bodies without possibilities of flexibility or of suppleness in the liaisons and the joints. But a such hypothesis is not realistic in numerical applications. A simple way to introduce suppleness in the joints of a articulated mechanic system is to place a small spring (fictive) massless in the liaisons between bodies as shown in the figure \ref{Fig:flexiart}. The spring exerts a restoring force on each bodies where the spring is attached. This force is applied on the fixation point of the spring, and is a monotonically increasing function of the spring elongation. This force is added on the other forces applied on the system in the writing of the movement equations. When the elasticity is thus introduced in a joint of two bodies of the system, It seems to be logical that where the constraints of corresponding liaisons disapear, and that the number of degrees of freedom is correlatively increased. The method is illustrated on a simple example of a double bodies system. 

\begin{exemple} {\bf Two bodies system with an elastic joint .}

\begin{figure}[ht]
\begin{center}
\includegraphics[width=3in]{flexiart}
\caption{Elastic joint modelisation }
\label{Fig:flexiart}
\end{center}
\end{figure}
Let's consider the two bodies system represented on the figure  \ref{Fig:flexiart}.
The movement equations of the two bodies are written as follow~:
\begin{align}
m_1\ddot x_1 &= F_1, \label{eqmou1}\\
m_1\ddot y_1 &= F_2, \\
I_1\ddot \theta_1 &= F_2d_1\cos\theta_1 - F_1d_1\sin\theta_1, \\
m_2\ddot x_2 &= - F_1, \\
m_2\ddot y_2 &= - F_2, \\
I_2\ddot \theta_2 &=  F_1d_2\sin\theta_2 - F_2d_2\cos\theta_2, \label{eqmou6}
\end{align}
where $F_1$ and $F_2$ indicate the amplitudes  of the components of the restoring forces applied on two bodies due to the presence of the spring. 

The cartesian coordinates of the fixation points of the spring on the two bodies are expressed like this :
\begin{align*}
\tilde x_1 &= x_1 + d_1\cos\theta_1, \\
\tilde x_2 &= x_2 - d_2\cos\theta_2, \\
\tilde y_1 &= y_1 + d_1\sin\theta_1, \\
\tilde y_2 &= y_2 - d_2\sin\theta_2. 
\end{align*}
The elongation of the spring is defined as the vector of components  $\epsilon_1$ and 
$\epsilon_2$ :
\eqnn
\epsilon_1 = \tilde{x}_2 - \tilde{x}_1 \hspace{10mm} \epsilon_2 = \tilde{y}_2 - \tilde{y}_1
\eeqnn
The restoring forces $F_1$ et $F_2$ are modeled as monotone increasing functions of composents of the elongation (see Figure \ref{Fig:forcelong})~:
\eqnn
F_1 = r(\epsilon_1) \hspace{10mm} F_2 = r(\epsilon_2)
\eeqnn

\begin{figure}[ht]
\begin{center}
\includegraphics[width=4cm]{forcelong}
\caption{elastic joints : restoring force with respect to the elongation }
\label{Fig:forcelong}
\end{center}
\end{figure}
Often, for some simplification reasons,the linear model is adopted, ie :
\begin{align*}
F_1 &= k_0(\tilde x_2 - \tilde x_1) = k_0((x_2 - x_1) - (d_1\cos\theta_1 + d_2\cos\theta_2)), \\
F_2 &= k_0(\tilde y_2 - \tilde y_1) = k_0((y_2 - y_1) - (d_1\sin\theta_1 + d_2\sin\theta_2)),
\end{align*}
where the constant $k_0$ is called {\em spring constant}. In this case, the movement equations  (\ref{eqmou1})-(\ref{eqmou6}) are rewritten as follows:
\begin{align*}
m_1\ddot x_1 &= k_0((x_2 - x_1) - (d_1\cos\theta_1 + d_2\cos\theta_2)), \\
m_1\ddot y_1 &= k_0((y_2 - y_1) - (d_1\sin\theta_1 + d_2\sin\theta_2)), \\
I_1\ddot \theta_1 &= k_0d_1((x_1 - x_2  + d_1\cos\theta_1 + d_2\cos\theta_2)\sin\theta_1 \\
&\hd +  (y_2 -y_1  - d_1\sin\theta_1 - d_2\sin\theta_2)\cos\theta_1), \\
m_2\ddot x_2 &= - k_0((x_2 - x_1) - (d_1\cos\theta_1 + d_2\cos\theta_2)), \\
m_2\ddot y_2 &= - k_0((y_2 - y_1) - (d_1\sin\theta_1 + d_2\sin\theta_2)), \\
I_2\ddot \theta_2 &=  k_0d_2((x_2 - x_1 - d_1\cos\theta_1 - d_2\cos\theta_2)\sin\theta_2 \\
&\hd +  (y_1 -y_2 + d_1\sin\theta_1 + d_2\sin\theta_2)\cos\theta_2). \qed
\end{align*}
\end{exemple}

This example shows that in the case of mechanic articulated system,the general dynamic model  (\ref{modmecgen}) is modified as :
\eqn
M(q)\ddot{q} + f(q,\dot{q}) + g(q) + k(q) = G(q)u \label{modgenflex}
\eeqn
where appears the additional term $k(q)$ representing the effect of restoring forces due to the presence of elastic joints in the system.

\section{Friction}
\markboth{{\bf \hspace*{5mm}Chapitre 2 }\hfill Articulated mechanical systems}
{{ \bf Sec. \thesection }\hfill Friction\hspace*{5mm}} 

The presence of friction forces is another physic phenomenon that was neglected until this moment and that is often an important effect on the movement of mechanic systems. In particular, in the case of elastic joints modeled like in the precedent section, the presence of an damping by the friction is essential to avoid the developing of models that would be the oscillations centers which are not really conform to the experiment reality.

There are many ways to introduce the friction in the description of a mechanic articulated system. We will retain here the most simple description supposing  that the movement of each coordinates  $q_i$ of the vector of generalized coordinates  $q = (q_1, q_2, \hdots , q_\delta)$ ,…) is affected by a separated friction force only depending of the speed  $(\dot{q_i})$ of this same coordinate and denoted $h_i(\dot{q_i})$. The vector of these friction forces is written   
$$
h(\dot{q}) = \bma{c} h_1(\dot{q_1}) \\ h_2(\dot{q_2}) \\ \vdots \\ h_{\delta}(\dot{q_\delta}) \ema
$$
in order that the general dynamic model (\ref{modgenflex}) is increased as follows  :
$$
M(q)\ddot{q} + f(q,\dot q) + g(q) + k(q) + h(\dot{q}) = G(q)u. \label{modgenflexfrot}
$$
The most common form of the functions $h_i(\dot q_i)$ est is this one :
$$
h_i(\dot q_i) = \alpha_{i}\mbox{sign}(\dot q_i) + \beta_i(\dot q_i).
$$
In this equation, the first term  $\alpha_{i}\mbox{sign}(\dot q_i)$ represents the dry friction while the second term $\beta_i(\dot q_i)$ represents the viscous friction. The $\alpha_i$ coefficient is constant. The function $\beta_i$ is monotonically increasing with $\beta(0) = 0$. It is noticed that the function $h$ is discontinuous at the origin, 
that can involve difficulties for the simulation and the analysis of the system. In the applications that will be considered in this book, except inverse indication, it is supposed that the dry friction is neglected ($\alpha_i = 0$).

\section{Energy and Euler-Lagrange equation} 

The kinetic energy $E_C$ of a articulated mechanic system is defined as ~:
$$
E_C(q,\dot q) =  \frac{1}{2} \dot q^T M(q) \dot q.
$$
The potential energy  $E_P$ is the primitive of the sum of forces deriving of a potential, to be more precise, gravity forces and restoring forces of springs ~:
$$
\frac{\partial E_P(q)}{\partial q} = g^T(q) + k^T(q).
$$
The total energy $E_T$ is the sum of the kinetic and potential energy :
$$
E_T = E_C + E_P.
$$
The evolution of the total energy during the movement of the system is examined by calculating his temporal derivative :
\begin{equation} \begin{split}
\dot E_T &= \frac{\partial E_C}{\partial \dot q}\ddot q + \frac{\partial E_C}{\partial q}\dot q
+ \frac{\partial E_P}{\partial q}\dot q \\
&= \dot q^T [M(q) \ddot q + \frac{1}{2} \dot M(q) \dot q + g(q) + k(q)].
\end{split} \end{equation}
By substituting the expression  $M(q)\ddot q$ extracted of the general equation of the movement 
(\ref{eqmouv}), this result is obtained :
$$
\dot E_T = \frac{1}{2} \dot q^T [\dot M(q) - 2C(q,\dot q)] \dot q + \dot q^T [G(q)u - h(\dot q)].
$$
The first term of the right member of this equation is equal to zero because the matrix  $\dot M(q) - 2C(q,\dot q)$
is anti-symmetrical (see above). It remains :
$$
\dot E_T = \dot q^T [G(q)u - h(\dot q)].
$$
When the system is not subjected by any other forces than forces deriving of a potential, the total energy is constant during the entire movement :
$$
G(q)u - h(\dot q) = 0 \hspace{5mm} \Rightarrow \hspace{5mm} \dot E_T = 0.
$$
In this case, It is said that the system is  {\em conservative }.

By using proprieties (\ref{prop1}) and (\ref{prop2}), it is verified that the kinetic energy satisfied this following relation:
\eqnn
\frac{d}{dt}(\frac{\partial E_C}{\partial \dot q})^T - (\frac{\partial E_C}{\partial q})^T = 
M(q)  \ddot q + C(q, \dot q) \dot q.
\eeqnn
this results in a alternative expression of the general movement equation (\ref{eqmouv})
is given by the expression :
\eqnn
\frac{d}{dt} (\frac{\partial L(q, \dot q)}{\partial \dot q})^T - (\frac{\partial L(q, \dot q)}
{\partial q})^T = G(q)u - h(\dot q)
\eeqnn
with :
\eqnn
L(q, \dot q) \triangleq E_C(q, \dot q) - E_P(q, \dot q).
\eeqnn
This equation is generally called the {\em Euler-Lagrange equation} and the quantity  $L(q, \dot q)$ is called {\em Lagrangien} of the system.

\section{Nonholonomic  system }

The nonholonomics  systems are articulated mechanic systems for which course constraints can non only depend  of positions 
$q$ but also speeds $\dot q$. When these constraints can not  be integrated to produce course constraints which depend exclusively of configuration coordinates, there are called {\it nonholonomic}. This situation is produced under various practical applications, more specifically in automotive, robotic and aviation industry.  We consider the particular case of one specific having $\delta$ degrees of freedom which is subjected to 
$m$ nonholomics independent constraints ($m < \delta$) which are linear with respect to speeds :
$$
N^T(q) \dot q = 0
$$
with the full row matrix $N^T(q)$ of dimensions $(m \times \delta)$. 
The matrix $S(q)$, of dimensions $\delta \times (\delta - m)$ and of full row is defined as :
\eqnn
N^T(q)S(q)=0.
\eeqnn
The constraints are equivalents to the fact that the vector of speeds 
$\dot q$ belongs to the space generated by the columns of the matrix 
$S(q)$ or, in another way,  there is a vector $\eta$ of dimension
$(\delta-m)$ as:
\eqn
\dot q = S(q) \eta. \label{modcin}
\eeqn
The movement equations are written under the standard form  :
\eqnn
M(q) \ddot{q} + f(q, \dot q) + g(q) + N(q)\lambda = G(q) u,
\eeqnn
by adding the term  $N(q)\lambda$ 
mouvement that represents the liaison forces which guarantee that the constraints are satisfied along the movement (see section 1.2.3). By suppressing the Lagrange multipliers On  $\lambda$ and by pre-multiplying this equation by $S^T(q)$:
\eqnn
S^T(q)M(q) \ddot{q} + S^T(q)[C(q, \dot q) \dot q + g(q)] = S^T(q)G(q)u.
\eeqnn
Finally, by using the relation (\ref{modcin}), this expression is obtained:
\eqn
J(q) \dot \eta + F(q, \eta) = S^T(q)G(q)u \label{moddyn}
\eeqn
avec
\eqnn
J(q) &=& S^T(q)M(q)S(q) \\
F(q, \eta) &=& S^T(q) M(q) \{ [\partial_q S(q)] S(q) \eta \} \eta +
S^T(q)f(q, S(q) \eta).
\eeqnn
The general dynamic model of a nonholonomic system consists in equation  (\ref{modcin}) and (\ref{moddyn}) that we can write under a state model:
\eqnn
\dot q &=& S(q) \eta \\
\dot \eta &=& J^{-1}(q) [ -F(q, \eta) + S^T(q)G(q) u ].
\eeqnn
We can observe that the state vector :
\eqnn
\bma{c} q \\ \eta \ema
\eeqnn
is of dimension $(2\delta-m)$ with coordinates $\eta$ homogeneous to speeds. 


\section{Exercices}

%\begin{exercice}{\bf \em Robots manipulateurs}
\begin{exercice}{\bf \em \blue{Technician} robots}

%On a représenté à la figure \ref{Fig:exe-robot} trois configurations de robots
%planaires à deux degrés de liberté.  Pour chacune de ces
%configurations~:
Figure \ref{Fig:exe-robot} represents tree configurations of planar robots. For each of this configurations~:
\begin{enumerate}
%\item Etablir le modèle dynamique du système et le modèle
%d'état correspondant.  Expliciter les matrices $M(q), C(q,\dot q)$
%et $G(q)$ ainsi que le vecteur $g(q)$.
\item Establish the dynamic model of the system and the corresponding state model. Make explicit the matrices $M(q), C(q,\dot q)$ and $G(q)$, and the vector $g(q)$.
%\item Vérifier que le modèle est conservatif et qu'il satisfait l'équation
%d'Euler-Lagrange.
\item Check that the model is conservative and that it satisfies Euler-Lagrange equation.
%\item Indiquer comment se modifient les équations du modèle si
%les segments sont soumis à un frottement visqueux proportionnel au
%carré de la vitesse. \qed
\item Indicate how the equations of the model are modified if the segments are subjected to a viscous friction proportional to the square of the speed. \qed 
\end{enumerate}
\begin{figure}[ht]
\begin{center}
\includegraphics[width=3.5in]{exe-robot}
%\caption{Configurations de robots manipulateurs planaires}
\caption{Configurations of \blue{technician} planar robots}
\label{Fig:exe-robot}
\end{center}
\end{figure}
\end{exercice}
\vv
\begin{figure}[!h]
\begin{center}
\includegraphics[width=8cm]{exe-fusee}
%\caption{Fusée à moteur orientable}
\caption{Rocket with \blue{adjustable} motor}
\label{Fig:exe-fusee}
\end{center}
\end{figure}
%\begin{exercice}{\bf \em Modélisation de la dynamique d'une fusée}
\begin{exercice}{\bf \em Modelling of the dynamics of a rocket}

%On considère une fusée propulsée par un moteur à réaction
%orientable comme indiqué sur la figure \ref{Fig:exe-fusee} et se dépla\c
%cant dans un plan vertical.  L'orientation du moteur est pilotée
%par un actionneur hydraulique fournissant un couple $T$.  Le moteur
%lui-même fournit une force de propulsion $F$.
Let's consider a rocket propelled by an \blue{adjustable} jet engine as indicated on figure \ref{Fig:exe-fusee} and moving in a vertical plane. The engine orientation is controlled by a hydraulic \blue{actuator} providing a torque $T$. The engine itself provides a propelling force $F$ .
\begin{enumerate}
%\item Etablir les équations du modèle d'état du système,
%sous l'hypothèse que les deux parties de la fusée (corps
%principal et moteur) sont des corps rigides de masse constante. \qed
\item Establish the equations of the state model of the system, assuming that the two parts of the rocket (main body and engine) are rigid bodies of constant mass. \qed
\end{enumerate}
\end{exercice}

%\begin{exercice}{\bf \em Modélisation dynamique du module d'excursion
%lunaire}
\begin{exercice}{\bf \em Dynamic modelling of the lunar excursion module}

%Lors de la mission Apollo 11, les astronautes Armstrong et Aldrin
%se sont posés sur la lune au moyen du LEM (Lunar
%Excursion Module; Fig. \ref{Fig:exe-lem}).  On considère les hypothèses de
%modélisation suivantes :
During the Apollo 11 mission, astronauts Armstrong and Aldrin landed on the moon using the LEM (Lunar Excursion Module; Fig. \ref{Fig:exe-lem}). We consider the following model assumptions : 
\begin{figure}[ht]
\begin{center}
\includegraphics[width=2in]{exe-lem}
%\caption{Module d'excursion lunaire}
\caption{Lunar Excursion Module}
\label{Fig:exe-lem}
\end{center}
\end{figure}
\begin{itemize}
%\item[a)] le LEM est un corps rigide
\item[a)] the LEM is a rigid body
%\item[b)] le mouvement est vertical
\item[b)] the movement is vertical
%\item[c)] les forces agissant sur le système sont la poussée $F$
%et l'attraction lunaire
\item[c)] the forces acting on the system are the thrust $F$ and the lunar attraction
%\item[d)] la masse de combustible embarqué constitue une partie
%importante (non négligeable) de la masse totale du LEM
\item[d)] the mass of fuel onboard is an
important (non negligible) part of the total mass of the LEM
%\item[e)] la masse de combustible consommée par unité de
%temps est proportionnelle à $F$.
\item[e)] the mass of fuel consumed per unit of time is proportional to $F$.
\end{itemize}
\begin{enumerate}
%\item Etablir un modèle d'état du système qui satisfait ces
%hypothèses de modélisation.
\item Establish a state model of the system that satisfies these
modelling assumptions.
%\item Quelles sont les principales limites de validité de ce
%modèle ? \qed
\item What are the main limits of validity of this
model ? \qed
\end{enumerate}
\end{exercice}
\vv

%\begin{exercice}{\bf \em Un train pendulaire}
\begin{exercice}{\bf \em A pendular train}

%Un train pendulaire est un train qui peut se déplacer à
%très grande vitesse dans les virages sans qu'il soit nécessaire
%d'incliner les voies. 
A pendular train is a train that can move at very high speed in bends without the need of inclining the rails.
\begin{figure}[ht]
\begin{center}
\includegraphics[width=7cm]{train2}
\includegraphics[width=5cm]{trainpendulaire}
%\caption{Un train pendulaire}
\caption{A pendular train}
\label{Fig:train}
\end{center}
\end{figure}
\begin{figure}[!h]
\begin{center}
\includegraphics[width=3in]{exe-camion}
%\caption{Modélisation de la dynamique d'un camion}
\caption{Modelling of the dynamics of a truck}
\label{Fig:exe-camion}
\end{center}
\end{figure}
%Pour cela chaque voiture est munie d'un dispositif
%actif qui applique une force verticale à la caisse de la voiture pour
%contrebalancer l'effet de la force \og centrifuge \fg. Ceci est illustré sur
%la figure \ref{Fig:train} où une section de la caisse d'une voiture est
%représentée schématiquement avec $F_g$ la force de gravité
%(appliquée au centre de masse $G$), $F_c$ la force "centrifuge" et
%$F_a$ la force appliquée. On suppose que la ligne d'action de la force
%$F_a$ est verticale quelle que soit la position angulaire $\theta$ de la
%caisse. D'autre part la suspension de la voiture est schématisée par
%un ressort vertical qui exerce une force proportionnelle à son
%élongation. Le point d'application
%$P$ du ressort est {\it contraint de se déplacer verticalement}.
%Etablir un modèle d'état du système. \qed
For this each car is equipped with an active device that applies a force vertical to the car \blue{body} to compensate the effect of the \og centrifugal \fg force. This is illustrated on
figure \ref{Fig:train} where a section of the \blue{body} of a car is schematically represented with  the gravity force $F_g $ (applied to the center of mass $G$) the "centrifugal" force $F_c$ and the applied force $F_a$. It is assumed that the action line of force $F_a$ is vertical irrespective of the angular position $\theta$ of the \blue{body}. On the other hand the car's suspension is represented by a vertical spring that exerts a force proportional to its elongation.  The application point $P$ of the spring is \it{forced to move vertically}. Established the state model of the system. \qed

\end{exercice}
\vv

%\begin{exercice}{\bf \em Modélisation de la dynamique d'un camion}
\begin{exercice}{\bf \em Modelling of the dynamics of a truck}

%On considère un camion se dépla\c cant en ligne droite (Fig. \ref{Fig:exe-camion}), sous les %hypothèses de modélisation suivantes~:
We consider a truck traveling in a straight line (Fig. \ref{Fig:exe-camion}), under the following modelling assumptions~:
\begin{itemize}
%\item[a)] Le camion est un système articulé composé de corps rigides (caisse et
%roues).
\item[a)] The truck is an articulated system composed of rigid bodies (\blue{body} and wheels)
%\item[b)] Le camion est équipé d'une propulsion arrière (le couple développé par
%le moteur est transmis aux roues arrières).
\item[b)] The truck is equipped with a rear propulsion (the torque developed by
the motor is transmitted to the rear wheels).
%\item[c)] Les roues roulent sans glisser.
\item[c)] The rear wheels roll without slipping.
%\item[d)] Les roues sont reliées au chassis par un système de suspension composé
%d'un ressort linéaire et d'un amortisseur à frottement visqueux de masse
%négligeable.  Ce système de suspension ne permet que des déplacements verticaux.
\item[d)] The wheels are connected to the frame by a suspension system composed of a linear spring and a viscous friction shock absorber of negligible mass. This suspension system allows only vertical displacements.
\end{itemize}
\begin{enumerate}
%\item Etablir un modèle d'état du système qui satisfait ces hypothèses de
%modélisation (se limiter à deux corps : le chassis et une roue motrice).
\item Establish a state model of the system that satisfies the modelling assumptions (be limited to two bodies : the frame and a driving wheel)
%\item Quelles sont les principales limites de validité de ce modèle ? \qed
\item What are the main limits of validity of this model ? \qed
\end{enumerate}
\end{exercice}
\vv

%\begin{exercice}{\bf \em Un bateau}
\begin{exercice}{\bf \em A boat}

%Un bateau muni d'un moteur orientable de type \og hors-bord \gf se déplace sur un fleuve comme %illustré à la figure \ref{bateau} (vue du dessus). Le fleuve est de largeur constante (= 2$L$). La %poussée du moteur est représentée par le vecteur de longueur $F$ (= grandeur de la force de %propulsion) et d'orientation $\beta$. Le bateau est aussi soumis à la force du courant du fleuve qui %est une fonction parabolique de l'abscisse $y$ : le courant est nul aux deux bords et maximum au %milieu du fleuve. Quand le moteur est à l'arrêt, le bateau est entraîné à la vitesse du courant par la %force de frottement de l'eau sur la coque. 
A boat equipped with an \blue{adjustable} motor of type \og speedboat \gf moves on a river as shown in figure \ref{bateau} (top view). The river is of constant width (= 2$L$). The engine thrust is shown by the vector of length $F$ (= size of the propulsion force) and of orientation $\beta$. The boat is also subject to the force of the current of the river which is a parabolic function of axis $y$ : the current is zero at both edges and maximal at the middle of the river. When the engine is stopped, the boat is driven at the speed of the current by the friction force of water on the hull .
\begin{enumerate}
%\item Etablir un modèle d'état du système. Pour simplifier, on peut supposer que :
\item Establish a state model of the system. For simplicity, we can assume that : 
\begin{itemize}
%\item[a)] le bateau est un corps rigide de masse constante;
\item[a)] the boat is a rigid body of constant mass;
%\item[b)] le plan d'eau est quasi-horizontal et la gravité n'influence pas le mouvement du bateau;
\item[b)] the stretch of water is almost horizontal and gravity does not influence the movement of the boat;
%\item[c)] la force exercée par le courant s'applique ponctuellement au centre de masse du bateau (on néglige le fait que la force du courant peut s'exercer de manière variable en divers points de la coque).
\item[c)] the force exerted by the current is locally applied at the center of mass of the boat (we neglect the fact that the current force can be exerted in variable ways in diverse points of the hull)
\end{itemize}
%\item Quelle doit être la capacité de propulsion du moteur pour que l'on ait la garantie que le bateau pourra remonter le courant ? \qed
\item What should be the thrust capacity of the engine to have the guarantee that the boad will be able to go against the current?
\end{enumerate} 
\begin{figure}[h]
\begin{center}
\includegraphics[width=10cm]{bateau}
%\caption{Un bateau}
\caption{A boat}
\label{bateau}
\end{center}
\end{figure}
\end{exercice}
\vv

\end{document}
